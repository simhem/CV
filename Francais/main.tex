%%%%%%%%%%%%%%%%%
% This is an sample CV template created using altacv.cls
% (v1.0.1, 11 September 2016) written by LianTze Lim (liantze@gmail.com).
% 
%% It may be distributed and/or modified under the
%% conditions of the LaTeX Project Public License, either version 1.3
%% of this license or (at your option) any later version.
%% The latest version of this license is in
%%    http://www.latex-project.org/lppl.txt
%% and version 1.3 or later is part of all distributions of LaTeX
%% version 2003/12/01 or later.
%%%%%%%%%%%%%%%%

%% If you need to pass whatever options to xcolor
\PassOptionsToPackage{dvipsnames}{xcolor}
\documentclass[10pt,a4paper]{altacv}

%% AltaCV uses the fontawesome and academicon fonts
%% and packages. 
%% See texdoc.net/pkg/fontawecome and http://texdoc.net/pkg/academicons for full list of symbols.
%% 
%% Compile with LuaLaTeX for best results. If you
%% want to use XeLaTeX, you'll need to install
%% Academicons.ttf in your operating system's font %% folder.


% Change the page layout if you need to
\geometry{left=1cm,right=9cm,marginparwidth=6.8cm,marginparsep=1.2cm,top=1.25cm,bottom=1.25cm}

% Change the font if you want to.
\setmainfont{Lato}

% Change the colours if you want to
\definecolor{Mulberry}{HTML}{72243D}
\definecolor{SlateGrey}{HTML}{2E2E2E}
\definecolor{LightGrey}{HTML}{666666}
\colorlet{heading}{Sepia}
\colorlet{accent}{Mulberry}
\colorlet{emphasis}{SlateGrey}
\colorlet{body}{LightGrey}

% Change the bullets for itemize and rating marker
% for \cvskill if you want to
\renewcommand{\itemmarker}{{\small\textbullet}}
\renewcommand{\ratingmarker}{\faCircle} 


\begin{document}
\name{Hemi Simon / DevOps}
\personalinfo{%
  % Not all of these are required!
  % You can add your own with \printinfo{symbol}{detail}
  \email{simon.hemi@gmail.com}
  \phone{+33637877136}
  \mailaddress{100 rue de la chapelle, 75018 Paris, France}
  \location{France}
  \linkedin{www.linkedin.com/in/simon-hemi-784a46b4}
  \age{25 ans}
  \versionsymbol{v2020m03}
}

%% Make the header extend all the way to the right, if you want. Extend the right margin by 8cm (=6.8cm marginparwidth + 1.2cm marginparsep)
\begin{adjustwidth}{}{-8cm}
\makecvheader
\end{adjustwidth}

\cvsection[page1sidebar]{Etudes et diplômes}

\cvevent{Diplôme d’ingénieur en génie informatique 
\newline \textit{\small{Filière système et réseaux} informatique}}{UTC - Université technologique de Compiègne}{Sept 2014 - Sept 2017}{Compiegne – France}
\divider

\cvevent{DUT informatique}{IUT Orsay}{Sept 2012 - Sept 2014}{Orsay – France}

%% Provide the file name containing the sidebar contents as an optional parameter to \cvsection.
%% You can always just use \marginpar{...} if you do
%% not need to align the top of the contents to any
%% \cvsection title in the "main" bar.
\cvsection{Expérience}

\cvevent{Responsable technique / DevOps}{Enedis / Devoteam}{Dec 2017 – Mar 2020 (2 ans 4 mois)}{Nanterre – France}
Responsable technique dans une equipe qui travaille sur un outil de test automatisés pour fonctionnels, de la RPA en python et via UiPath et des bouchons intelligents en python :
\medskip
\begin{itemize}
    \item Déploiement automatisé d'infrastructures et d'applicatifs
    \item Implementation d'application web
    \item Installations et mises à jour de dépendances/d'outillage 
    \item Mise en place de monitoring, de sauvegarde et de nettoyage de l'infrastructure  
    \item Maintien et montée de version du socle de tests automatisés 
    \item Support technique / Veille technologique
\end{itemize}

\divider

\cvevent{Stage : Software development}{Siemens Corporate Technology}{Fevr 2017 – Aout 2017 (6 mois)}{Princeton – Etats-Unis}
\begin{itemize}
\item Analyse/Développement de logiciels (\textbf{C\# / C++})
\end{itemize}

\divider

\cvevent{Stage : développement d’un FrameWorkHTML5}{Aubay}{Sept 2015 – Fevr 2016 (6 mois)}{Boulogne-Billancourt – France}
\begin{itemize}
\item Développement de chaîne d'intégration continue (\textbf{NodeJs}) 
\item Etude et développement d'un Framework selon les nouveaux standards du web (\textbf{JS ES6 / HTML / CSS})
\end{itemize}

\divider

\cvevent{Association : Junior Ingénieur d'affaire}{USEC, Junior entreprise}{Fevr 2015 – Sept 2015 (6 mois)}{Compiegne - France}
\begin{itemize}
\item Gestion de projet \& relations client (\textbf{Agile})
\end{itemize}

\medskip

\clearpage

\end{document}
